\section{Object reconstruction}
\label{s:objetcs}

The final states of interest in this analysis contain electrons, muons, jets,
$b$-jets and missing transverse momentum.

Electron candidates~\cite{PERF-2013-03} are reconstructed from energy deposits
(clusters) in the EM calorimeter that are associated with reconstructed tracks
in the inner detector.  The electron identification relies on a
likelihood-based selection~\cite{ATLAS-CONF-2014-032,
ATL-PHYS-PUB-2015-041}. Electrons are required to pass the `medium'
likelihood identification requirements described in Ref.~\cite{ATL-PHYS-PUB-2015-041}. These include requirements on the 
shapes of the electromagnetic shower in the calorimeter as well as tracking and track-to-cluster matching quantities. 
The electrons are also required to have transverse momentum $\pt > 7\,\GeV$ and
$|\eta_\text{cluster}| < 2.47$, where $\eta_\text{cluster}$ is the
pseudorapidity of the calorimeter energy deposit associated with the electron
candidate.  Candidates in the EM calorimeter barrel/endcap transition region
$1.37 < |\eta_\text{cluster}| < 1.52$ are excluded.  

Muon candidates are reconstructed from a fit to track segments in the various layers of
the muon spectro\-meter, matched with tracks identified in the inner detector.
Muons are required to have $\pt > 7\,\GeV$ and $\abseta < 2.4$ and to pass the
`medium' identification requirements defined in Ref.~\cite{PERF-2015-10}. The `medium' requirement includes selections on the numbers of hits
in the ID and MS as well as a compatibility requirement between momentum measurements in the ID and MS. It provides a high efficiency and purity of selected muons. Electron candidates sharing a track with a muon candidate are removed.  

To reduce the non-prompt lepton background from hadron decays or jets
misidentified as leptons (labelled as ``fake leptons'' throughout this paper), electron and muon
candidates are required to be isolated. The total sum of track transverse
momenta in a surrounding cone of size $\min(10\,\GeV/\pT, r_{e,\mu})$, excluding the track of the candidate from the sum, is required to be less than 6\% of the candidate \pt, where
$r_e = 0.2$ and $r_{\mu} = 0.3$.  In addition, the sum of the cluster
transverse energies in the calorimeter within a cone of size $\Delta R_{\eta} \equiv
\sqrt{(\Delta\eta)^2 + (\Delta\phi)^2} = 0.2$ of any electron candidate,
excluding energy deposits of the candidate itself, is required to be less than
6\% of the candidate \pt.

For both electrons and muons, the longitudinal impact parameter of the
associated track with respect to the primary vertex,\footnote{A primary vertex
candidate is defined as a vertex with at least five associated tracks,
consistent with the beam collision region.  If more than one such vertex is
found, the vertex candidate with the largest sum of squared transverse momenta
of its associated tracks is taken as the primary vertex.} $z_{0}$, is required
to satisfy $|z_0 \sin\theta|<0.5$ mm. The significance of the transverse impact
parameter $d_0$ is required to satisfy $|d_0|/\sigma(d_0)<5$ for electrons and
$|d_0|/\sigma(d_0)<3$ for muons, where $\sigma(d_0)$ is the uncertainty in
$d_0$.

Jets are reconstructed using the anti-$k_t$ algorithm~\cite{Cacciari:2008gp,
Cacciari:2005hq} with radius parameter $R = 0.4$, starting from topological
clusters in the calorimeters~\cite{Aad:2016upy}. The effect of pile-up on jet
energies is accounted for by a jet-area-based correction~\cite{Cacciari:2008gn}
and the energy resolution of the jets is improved by using global sequential
corrections~\cite{ATLAS-CONF-2015-002}. Jets are calibrated to the hadronic
energy scale using $E$- and $\eta$-dependent calibration factors based on MC
simulations, with in-situ corrections based on Run-1 data~\cite{PERF-2011-03,
ATLAS-CONF-2015-037} and checked with early Run-2
data~\cite{ATL-PHYS-PUB-2015-015}.  Jets are accepted if they fulfil the
requirements $\pT > 25\,\GeV$ and $|\eta| < 2.5$. To reduce the contribution
from jets associated with pile-up, jets with $\pT < 60\,\GeV$ and $|\eta| < 2.4$
are required to satisfy pile-up rejection criteria (JVT), based on a multivariate
combination of track-based variables ~\cite{PERF-2014-03}. 


%\cite{ATLAS-CONF-2014-018}.

Jets are $b$-tagged as likely to contain $b$-hadrons using the \tool{MV2c20}
algorithm, a multivariate discriminant making use of the long lifetime, large
decay multiplicity, hard fragmentation and high mass of $b$-hadrons
\cite{PERF-2012-04}.  The average efficiency to correctly tag a $b$-jet is
approximately $77\%$, as determined in simulated \ttbar events, but it varies as
a function of \pT and $\eta$.  In simulation, the tagging algorithm gives a
rejection factor of about $130$ against light-quark and gluon jets, and about
$4.5$ against jets containing charm quarks~\cite{ATL-PHYS-PUB-2015-022}.  The
efficiency of $b$-tagging in simulation is corrected to that in data using a
\ttbar-based calibration using Run-1 data~\cite{ATLAS-CONF-2014-004} and
validated with Run-2 data~\cite{ATL-PHYS-PUB-2015-039}.

The missing transverse momentum $\mathbf{p}^\text{miss}_\text{T}$, with
magnitude \met, is a measure of the transverse momentum imbalance due to
particles escaping detection. It is computed~\cite{PERF-2011-07} as the
negative sum of the transverse momenta of all electrons, muons and jets and an
additional soft term. The soft term is constructed from all tracks that are associated 
with the primary vertex but not with any physics object. In this way, the \met is adjusted for the best calibration of the jets
and the other identified physics objects above, while maintaining pile-up
independence in the soft term~\cite{ATL-PHYS-PUB-2015-027,
ATL-PHYS-PUB-2015-023}.

To prevent double-counting of electron energy deposits as jets, the closest jet
within $\Delta R_y = 0.2$ of a reconstructed electron is removed, where $\Delta
R_y \equiv \sqrt{(\Delta y)^2 + (\Delta\phi)^2}$.  If the nearest jet surviving
the above selection is within $\Delta R_y = 0.4$ of an electron, the electron
is discarded to ensure that selected electrons are sufficiently separated from
nearby jet activity.  To reduce the background from muons originating from heavy-flavour particle
decays inside jets, muons are removed if they are separated from the nearest
jet by $\Delta R_y < 0.4$.  However, if this jet has fewer than three
associated tracks, the muon is kept and the jet is removed instead; this avoids
an inefficiency for high-energy muons undergoing significant energy loss in the
calorimeter.
