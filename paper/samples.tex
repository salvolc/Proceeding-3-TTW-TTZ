\section{Data and simulated event samples}
\label{s:samples}

The data were collected with the ATLAS detector during 2015 with a 
bunch spacing of \SI{25}{ns} and a mean number of $14$ $pp$ 
interactions per bunch crossing (pile-up).
With strict data-quality requirements, the integrated luminosity considered
corresponds to \lumi with an uncertainty of $2.1\%$~\cite{Aaboud:2016hhf}. 

Monte Carlo simulation samples (MC) are used to model the expected signal and
background distributions in the different control, validation and signal regions
described below. The
heavy-flavour decays involving $b$- and $c$-quarks, 
particularly important to this measurement, are modelled
using the \EVTGEN~\cite{EvtGen} program, except for processes modelled using the
\SHERPA generator. In all samples the top-quark mass is set to $172.5\,\GeV$ and
the Higgs boson mass is set to $125\,\GeV$. The response of the detector to
stable\footnote{A particle is considered stable if $c\tau \ge 1$ cm.} particles
is emulated by a dedicated simulation~\cite{SOFT-2010-01} based either fully on
\GEANT~\cite{geant} or on a faster parameterisation~\cite{ATL-PHYS-PUB-2010-013}
for the calorimeter response and \GEANT for other detector systems. To account for
additional $pp$ interactions from the same and close-by bunch crossings, a set
of minimum-bias interactions generated using \PYTHIA v8.210~\cite{Sjostrand:2014zea}, 
referred to as \PYTHIA 8 in the following, 
with the \tune{A2}~\cite{ATL-PHYS-PUB-2011-014} set of tuned MC parameters (\tune{A2} tune)
is superimposed on the
hard-scattering events.  In order to reproduce the same pile-up levels present in
the data, the distribution of the number of additional $pp$ interactions in the
MC samples is reweighted to match the one in the data. All samples are processed
through the same reconstruction software as the data. Simulated events are
corrected so that the object identification, reconstruction and trigger
efficiencies, energy scales and energy resolutions match those determined from
data control samples.

The associated production of a top-quark pair with one or two vector bosons is
generated at leading order (LO) with \MGAMC\ interfaced to \PYTHIA 8, with up to
two (\ttW), one (\ttZ) or no ($\bgttWW$) extra partons included in the matrix
elements. The $\gamma^{*}$ contribution and the \Zgamstar interference are 
included in the \ttZ samples. 
The \tune{A14} ~\cite{ATL-PHYS-PUB-2014-021} set of tuned MC parameters (\tune{A14} tune) is used together with the 
\pdf{NNPDF2\!.\!3LO} parton distribution function (PDF) set~\cite{Ball:2012cx}.
The samples are normalised using cross sections computed at
NLO in QCD~\cite{ATL-PHYS-PUB-2016-005}.

The $t$-channel production of a single top quark in association with a $Z$ boson
(\tZ) is generated using \MGAMC\ interfaced with \PYTHIA v6.427~\cite{PythiaManual}, 
referred to as \PYTHIA 6 in the following, 
with the \pdf{CTEQ6L1} PDF~\cite{cteq6} set and the \tune{Perugia2012}~\cite{Skands:2010ak} 
set of tuned MC parameters at NLO in QCD. The \Zgamstar interference is
included, and the four-flavour scheme is used in the computation. 

The \Wt-channel production of a single top quark together with a $Z$ boson
(\WtZ) is generated with  \MGAMC\ and showered with \PYTHIA 8, using the 
\pdf{NNPDF3\!.\!0NLO} PDF set~\cite{Ball:2015cx} and the \tune{A14} tune. The generation is performed at
NLO in QCD using the five-flavour scheme. 
Diagrams containing a top-quark pair are removed to avoid 
overlap with the \ttZ process.

Diboson processes with four charged leptons ($4\ell$), three charged leptons and
one neutrino ($\ell\ell\ell\nu$) or two charged leptons and two neutrinos
($\ell\ell\nu\nu$) are simulated using the \SHERPA 2.1
generator~\cite{Gleisberg:2008ta}. 
The matrix elements include all diagrams with four electroweak vertices. 
They are calculated for up to one ($4\ell,
\ell\ell\nu\nu$) or no additional partons ($\ell\ell\ell\nu$) at NLO and up to three
partons at LO using the \COMIX~\cite{Gleisberg:2008fv} and
\OPENLOOPS~\cite{Cascioli:2011va} matrix element generators and merged with the
\SHERPA parton shower using the \textsc{ME+PS@NLO}
prescription~\cite{Hoeche:2012yf}. The \pdf{CT10nlo} PDF set~\cite{Lai:2010vv} is used in conjunction
with a dedicated parton-shower tuning developed by the \SHERPA authors. The NLO
cross sections calculated by the generator are used to normalise diboson
processes.  Alternative diboson samples are simulated using the \POWHEGBOX
v2~\cite{Powbox2} generator, interfaced to the \PYTHIA 8 parton shower model,
and for which the \pdf{CT10nlo} PDF set is used in the matrix element, while the
\pdf{CTEQ6L1} PDF set is used for the parton shower along with the
\tune{AZNLO}~\cite{AZNLO:2014} set of tuned MC parameters.

The production of three massive vector bosons with subsequent leptonic decays of
all three bosons is modelled at LO with the \SHERPA 2.1 generator and the
\pdf{CT10} PDF set~\cite{Lai:2010vv}. Up to two additional partons are included in the matrix
element at LO and the full NLO accuracy is used for the inclusive process.  

Electroweak processes involving the vector-boson scattering (VBS) diagram and
producing two same-sign leptons,  two neutrinos and two partons are modelled
using \SHERPA 2.1 at LO accuracy and the \pdf{CT10} PDF set. Processes of
orders four and six in the electroweak coupling constant are considered, and up
to one additional parton is included in the matrix element. 

For the generation of \ttbar events and \Wt-channel single-top-quark events the
\POWHEGBOX v2 generator is used with the \pdf{CT10} PDF set. The parton shower
and the underlying event are simulated using \PYTHIA 6
with the \pdf{CTEQ6L1} PDF set and the corresponding
\tune{Perugia2012} tune.  The \ttbar samples are normalised to their
next-to-next-to-leading-order (NNLO) cross-section predictions, including soft-gluon
resummation to next-to-next-to-leading-log order, as calculated with the
\tool{Top++2.0} program (see Ref.~\cite{ref:xs6} and references therein).  For more
efficient sample generation, the \ttbar sample is produced by selecting only true
dilepton events in the final state.  Moreover, an additional dilepton \ttbar
sample requiring a $b$-hadron not coming from top-quark decays is generated
after $b$-jet selection. Diagram removal is employed to remove
the overlap between \ttbar and \Wt~\cite{Re:2010bp}.

Samples of \ttbar events produced in association with a Higgs boson (\ttH) 
are generated using NLO matrix elements in \MGAMC\ with
the \pdf{CT10NLO} PDF set and interfaced with \PYTHIA 8 for the modelling of the
parton shower.  Higgs boson production via gluon--gluon fusion (ggF) and vector
boson fusion (VBF) is generated using the \POWHEGBOX v2 generator with
\pdf{CT10} PDF set. The parton shower and underlying event are simulated using
\PYTHIA 8 with the \pdf{CTEQ6L1} PDF set and \tune{AZNLO} tune.  Higgs boson
production with a vector boson is generated at LO using \PYTHIA 8 with the
\pdf{CTEQ6L1} PDF.  All Higgs boson samples are normalised using theoretical
calculations of Ref.~\cite{lhcxs}. 

Events containing $Z$ or $W$ bosons with associated jets, referred to as 
$Z$+jets and $W$+jets in the following, are simulated using the
\SHERPA 2.1 generator. Matrix elements are calculated for up to two partons at
NLO and four partons at LO.  The \pdf{CT10} PDF set is used in conjunction with
a dedicated parton-shower tuning developed by the \SHERPA authors~\cite{Gleisberg:2008ta}.  The
$Z/W$+jets samples are normalised to the NNLO cross
sections~\cite{Anastasiou:2003ds, Gavin:2010az, Gavin:2012sy, Li:2012wna}.
%~\cite{STDM-2011-06, ATLAS-CONF-2015-039}. 
Alternative $Z/W$+jets samples
are simulated using \MGAMC\ at LO interfaced to the \PYTHIA 8 parton shower
model. The \tune{A14} tune is used together with the \pdf{NNPDF2.3LO} PDF set. 

The SM production of three and four top quarks is generated at LO with
\MGAMC+\PYTHIA 8, using the \tune{A14} tune together with the
\pdf{NNPDF2\!.\!3LO} PDF set. The samples are normalised using cross
sections computed at NLO~\cite{Barger:2010uw,Bevilacqua:2012em}.
