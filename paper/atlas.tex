\section{The ATLAS detector}
\label{s:atlas}

The ATLAS detector~\cite{PERF-2007-01} consists of four main subsystems: an
inner tracking system, electromagnetic (EM) and hadronic calorimeters, and a
muon spectrometer (MS).  The inner detector (ID) consists of a high-gra\-nu\-la\-ri\-ty
silicon pixel detector, including the newly installed Insertable
B-Layer~\cite{ATL-INDET-PUB-2015-001}, which is the innermost layer of the
tracking system, and a silicon microstrip tracker, together providing precision
tracking in the pseudorapidity\footnote{ATLAS uses a right-handed coordinate
system with its origin at the nominal interaction point (IP) in the centre of
the detector and the $z$-axis along the beam pipe. The $x$-axis points from the
IP to the centre of the LHC ring, and the $y$-axis points upward. Cylindrical
coordinates $(r,\phi)$ are used in the transverse plane, $\phi$ being the
azimuthal angle around the $z$-axis. The pseudorapidity is defined in terms of
the polar angle $\theta$ as $\eta=-\ln\tan(\theta/2)$.} range $|\eta|<2.5$ and
of a transition radiation tracker covering $|\eta|<2.0$. All the systems are immersed in a
\SI{2}{T} magnetic field provided by a superconducting solenoid.  The EM
sampling calorimeter uses lead and liquid argon (LAr) and is divided into
barrel ($|\eta|<1.475$) and endcap ($1.375<|\eta|<3.2$) regions.  Hadron
calorimetry is provided by a steel/scintillator-tile calorimeter, segmented
into three barrel structures, in the range $|\eta|<1.7$, and by two copper/LAr
hadronic endcap calorimeters that cover the region $1.5<|\eta|<3.2$. The solid
angle coverage is completed with forward copper/LAr and tungsten/LAr
calorimeter modules, optimised for EM and hadronic measurements respectively,
covering the region $3.1<|\eta|<4.9$. The muon spectrometer measures the
deflection of muon tracks in the range $|\eta|<2.7$ using multiple layers of
high-precision tracking chambers located in toroidal magnetic fields.
The field integral of the toroids ranges between $2.0$ and \SI{6.0}{Tm} for
most of the detector.  The muon spectrometer is also instrumented with separate
trigger chambers covering $|\eta|<2.4$.  A two-level trigger system, using
custom hardware followed by a software-based trigger level, is used to reduce the event
rate to an average of around \SI{1}{kHz} for offline storage. 
