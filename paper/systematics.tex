\section{Systematic uncertainties}
\label{s:systematics}

The normalisation of signal and background in each channel can be affected by
several sources of systematic uncertainty. These are described in the
following subsections.

\subsection{Luminosity}
\label{sec:syst_lumi}
The uncertainty in the integrated luminosity in the 2015 dataset is 2.1\%. It is
derived, following a methodology similar to that detailed in
Ref.~\cite{DAPR-2011-01}, from a calibration of the luminosity
scale using $x$--$y$ beam-separation scans performed in August 2015. This
systematic uncertainty is applied to all processes modelled using \MC\
simulations.

\subsection{Uncertainties associated with reconstructed objects}
\label{sec:syst_objects}
 
Uncertainties associated with the lepton selection arise from imperfect
knowledge of the trigger, reconstruction, identification and isolation
efficiencies, and lepton momentum scale and resolution \cite{PERF-2013-03,
ATL-PHYS-PUB-2011-006, ATLAS-CONF-2014-032, ATL-PHYS-PUB-2015-041,
PERF-2015-10}.  The uncertainty in the electron identification
efficiency is the largest systematic uncertainty in the \TLC\ and among the
most important ones in the \FLC.    

Uncertainties associated with the jet selection arise from the jet energy scale
(JES), the JVT requirement and the jet energy resolution (JER).
Their estimations are based on Run-1 data and checked with early
Run-2 data.  The JES and its uncertainty are derived by combining information from
test-beam data, collision data and simulation~\cite{PERF-2012-01}.  JES
uncertainty components arising from the in-situ calibration and the jet flavour
composition are among the dominant uncertainties in the \SSLSR\ and \TL\
channels.  The uncertainties in the JER and JVT have a significant effect at
low jet \pt.  The JER uncertainty results in the second largest uncertainty in the
\TLC.

The efficiency of the flavour-tagging algorithm is measured for each jet
flavour using control samples in data and in simulation. From these
measurements, correction factors are defined to correct the tagging rates in
the simulation.  In the case of $b$-jets, correction factors and their
uncertainties are estimated based on observed and simulated $b$-tagging rates
in \ttbar dilepton events~\cite{ATLAS-CONF-2014-004}.  In the case of $c$-jets,
they are derived based on jets with identified $D^{*}$
mesons~\cite{ATLAS-CONF-2014-046}.  In the case of light-flavour jets,
correction factors are derived using dijet events~\cite{ATLAS-CONF-2014-046}.
Sources of uncertainty affecting the $b$- and $c$-tagging efficiencies are
considered as a function of jet \pt, including bin-to-bin
correlations~\cite{ATLAS-CONF-2014-004}.  An additional uncertainty is assigned
to account for the extrapolation of the $b$-tagging efficiency measurement from
the \pT region used to determine the scale factors to regions with higher \pT.
For the efficiency to tag light-flavour jets, the dependence of the uncertainty on the jet \pt and $\eta$ is considered.  These systematic uncertainties are taken as
uncorrelated between $b$-jets, $c$-jets, and light-flavour jets. 

The treatment of the uncertainties associated with reconstructed objects is common to all
three channels, and thus these are considered as correlated among different
regions.

\subsection{Uncertainties in signal modelling} 
\label{sec:signal_modeling}

From the nominal \MGAMC+\PYTHIA~8 (\tune{A14} tune) configuration, two
parameters are varied to investigate uncertainties from the modelling of the
\ttZ and \ttW processes: the renormalisation ($\mu_{\rm R}$) and factorisation ($\mu_{\rm F}$) scales.
A simultaneous variation of $\mu_{\rm R} = \mu_{\rm F}$ by factors $2.0$ and $0.5$ is performed.  In addition, the effects of a set of variations in the tune parameters
(\tune{A14} eigentune variations), sensitive to initial- and final-state
radiation, multiple parton interactions and colour reconnection, are evaluated.
Studies performed at particle level show that the largest impact comes from
variations in initial-state radiation~\cite{ATL-PHYS-PUB-2016-005}.  The
systematic uncertainty due to the choice of generator for the \ttZ and \ttW signals
is estimated by comparing the nominal sample with one generated with \SHERPA
v2.2. The \SHERPA sample uses the LO matrix element with up to one (two) additional parton(s) included
in the matrix element calculation for \ttZ  (\ttW) and merged with the \SHERPA
parton shower~\cite{Schumann:2007mg}  using the \textsc{ME+PS@LO} prescription.
The \pdf{NNPDF3\!.\!0NLO} PDF set is used in conjunction
with a dedicated parton shower tune developed by the \SHERPA authors.  Signal
modelling uncertainties are treated as correlated among channels.

\subsection{Uncertainties in background modelling}
\label{sec:bkg_modeling}

In the \TL\ and \SSLSR\ channels, the diboson background is dominated by $WZ$
production, while $ZZ$ production is dominant in the \FLC.  While the inclusive
cross sections for these processes are known to better than 10\%, they
contribute to the background in these channels if additional $b$-jets and 
other jets are produced and thus have a significantly larger uncertainty.

In the \TL\ and \SSLSR\ channels, the normalisation of the $WZ$ background is
treated as a free parameter in the fit used to extract the \ttZ and \ttW signals. The
uncertainty in the extrapolation of the $WZ$ background estimate from the
control region to signal regions with specific jet and $b$-tag multiplicities
is evaluated by comparing predictions obtained by varying the renormalisation,
factorisation and resummation scales used in MC generation. The
uncertainties vary across the different regions and an overall uncertainty of
$-50\%$ and $+100\%$ is used.

The normalisation of the $ZZ$ background is treated as a free parameter in the
fit used to extract the \ttZ and \ttW signals.  In the \FLC, several uncertainties in
the $ZZ$ background estimate are considered.  They arise from the extrapolation
from the \FLCR\ control region (corresponding to on-shell $ZZ$ production) to
the signal region (with off-shell $ZZ$ background) and from the extrapolation
from the control region without jets to the signal region with at least one
jet. They are found to be 30\% and 20\%, respectively.  An additional
uncertainty of 10--30\% is assigned to the normalisation of the heavy-flavour
content of the $ZZ$ background, based on a data-to-simulation comparison of
events with one $Z$ boson and additional jets and cross-checked with a
comparison between different $ZZ$ simulations~\cite{TOPQ-2013-05}.

The uncertainty in the \ttH background is evaluated by varying the factorisation and renormalisation scales up and down by a factor of two with respect to the nominal value, $H_{\rm T}/2$, where $H_{\rm T}$ is defined as the scalar sum of the transverse masses $\sqrt{\pt^2+m^2}$ of all final state particles.

For the \tZ background, an overall normalisation uncertainty of 50\% is
assumed.  An additional uncertainty affecting the distribution of this
background as a function of jet and $b$-jet multiplicity is evaluated by
varying the factorisation and renormalisation scales, as well as the amount of
radiation in the \tune{Perugia2012} parton shower tune.

An uncertainty of $+10\%$ and $-22\%$ is assigned to the \WtZ background cross
section. The uncertainty is asymmetric due to an alternative estimate of the
interference effect between this process and the \ttZ production.  The shape
uncertainty is evaluated by varying the factorisation and renormalisation
scales up and down by a factor of two with respect to the nominal value $H_{\rm T}/2.$

For other prompt-lepton backgrounds, uncertainties of 20\% are assigned to the
normalisations of the $WH$ and $ZH$ processes, based on calculations from
Ref.~\cite{Heinemeyer:2013tqa}.  An uncertainty of 50\% is considered for
triboson and same-sign $WW$ processes.

The fake-lepton background uncertainty is evaluated as follows.  The uncertainty due to the matrix method is estimated by propagating the
statistical uncertainty on the measurement of the fake-lepton efficiencies.
Additionally, a 20\% uncertainty is added to the subtracted charge-flip yields
estimated as the difference between data-driven charge-flips and simulation,
and the \met requirement used to enhance the single-fake-lepton fraction is varied by 
$20\,\GeV$. The main sources 
of fake muons are
decays of light-flavour or heavy-flavour hadrons inside jets.  For the \SSLSR\
region, the flavour composition of the jets faking leptons is assumed to be
unknown. To cover this uncertainty, the central values of the fake-lepton
efficiencies extracted from the $b$-veto and the $b$-tag control regions are
used, with the efficiency difference assigned as an extra uncertainty. For the
\FL\ channel, fake-lepton systematic uncertainties are covered by the
scale-factor uncertainties used to calibrate the simulated fake-lepton yield in
the control regions. Within a fake-lepton estimation method, all systematic
uncertainties are considered to be correlated among analysis channels and
regions. Thus \SSLSR\ and \TL\ fake-lepton systematic uncertainties that use the
matrix method are not correlated with the \FL\ systematic uncertainties.  The
expected uncertainties in the fake-lepton backgrounds relative to the total
backgrounds vary in each channel and signal region: 50\% for the \SSLSR\
region, 25--50\% for the \TLC\ and 5--10\% for the \FLC.
